\documentclass[10pt]{article}
\usepackage[utf8]{inputenc}
\usepackage[left=0.75in, right=0.75in, top=0.75in, bottom=0.75in]{geometry}
\title{Problem Statement}
\author{Leif Tsang }
\date{October 2018}
\date{CS461 Capstone Fall}

\usepackage{natbib}
\usepackage{graphicx}

\begin{document}
\fontfamily{cmss}

\maketitle
\vspace{3.5in}

\section{Abstract}
\sffamily
In the University Student Launch Initiative (USLI) we will be taking part in a NASA lead competition. This competition requires us to create a rocket and simulate real-life challenges a rocket might face when landing on a foreign planet. Our team’s performance will be evaluated and graded based on many different areas that have yet to be specified. The rocket will be an approximate of 10 feet long and will be launched a mile off the ground. Once it has reached the apex of its trajectory, we will be opening a parachute equipped within the rocket to ensure a soft landing. When touch down is complete we will remotely deploy a rover that will autonomously collect a soil sample to then be tested. This task has many components and variables that must be that must be functioning at full capacity in preparation for this competition. Extensive testing must be done on each part separately and as we get closer to the finished project, we will need to put the systems together for integrated testing. The final launch needs to be flawless and executed perfectly upon entering the competition. 

\vspace{1.5in}
\section{What is the problem?}
When launching rockets, there are very few humans to carry out tasks from within the vessel. It is very difficult for small tasks to be completed in space once on board the rocket. In order to accomplish tasks in safely, we rely on the work of robots or artificial intelligence. This competition has been created by NASA in order to test the skills of many different groups. We will be challenged to launch a rocket at a specified distance from the ground. Once the rocket reaches a certain height it will deploy its chute and land. After landing this rocket will deploy a rover which must be automated to drive around and obtain a soil sample from the new environment. The judges will be scoring us on our ability to use the soil sample to collect information as well as our execution and safety during the day of our rocket launch. Members of our team will be visiting schools to share this information with other generations. Ultimately we hope to inspire others and educate them on the benefits and vast opportunities that engineering can provide. 

\section{What is our solution?}
To complete NASA’s challenge we are designing and deploying a small scale robot upon the rocket’s landing. This robot will be able to drive using a collision detection software that we will create. Our software will make use of various different types of sensors. These will help steer our rover clear of any obstacles in our path. Since we have no control of where the rocket lands, we will need to plan for the worst case scenarios. The various scenarios and obstacles that we may face include terrain, weather, location and other potential contingencies. Our rover will be equipped with sonar and wheels of the maximum allowable size (about the inside diameter of the rocket) for the best clearance possible. The rover will only have 2 wheels and have a short body with a long axle between the wheels. This will prevent the rover from tipping over given rough and uneven terrain. This is also the most optimal vehicle shape to fit inside a rocket with correct orientation after being ejected from the rocket. The soil collection will be one of two options. We could implement a clam gun type technology that will drill into the ground and pull out a cylindrical chunk of dirt, or we could use an auger with a housing for the soil sample obtained by the drilling.

While in the air, the rocket is going to have many sensors that will accurately gauge its location along with altitude so that the parachute will deploy at the correct time and be traced to where the rover is ejected. We are also planning to equip the rocket with a gopro camera to gain footage during the entire rocket launch. With this footage, we hope to gain knowledge about the launch, as well as inspire others to share in our passion. 

\section{When will we know we are finished?}
Finally, after we feel confident we have a solid product that will achieve its goal, we will launch our rocket for a test run. During these launches there are many things that can go wrong. To list a few, the rocket may tumble, the parachute may get caught or not eject at all, the rover may not eject or move after it has ejected, the list goes on. Leading up to this final launch we will have numerous tests ranging from micro operations such as gathering soil, to obstacle courses testing our rovers maneuverability, up to rocket launches. We are confident that during our final launch, if everything goes according to plan and analysis of the aftermath goes well, then we will be ready for the upcoming competition.

\end{document}
